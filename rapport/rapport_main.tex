\documentclass[a4paper,french,towsides,10pt]{book}
\usepackage[utf8]{inputenc}
\usepackage[french]{babel}
\usepackage{fancyhdr}
\usepackage{enumerate}
\usepackage{graphicx}
\usepackage[french]{minitoc}
\usepackage{multirow}
\usepackage{placeins}
\usepackage{listings}
\usepackage{color}
\usepackage{float}
\usepackage{tabularx}
\usepackage[french,ruled,vlined,linesnumbered]{algorithm2e}
\usepackage{amsmath}
\usepackage{amssymb}
\usepackage[bookmarks=true]{hyperref}

\hypersetup{pdfborder={0 0 0}}
\pagestyle{fancy}
\setlength{\parskip}{1.5ex plus .4ex minus .4ex}
\renewcommand{\labelitemi}{\textbullet}
\renewcommand{\chaptermark}[1]{\markboth{#1}{}}

\pagestyle{fancy}

\renewcommand{\chaptermark}[1]{\markboth{#1}{}}
\renewcommand{\sectionmark}[1]{\markright{\thesection\ #1}}

\fancyhf{}

\fancyhead[RO,LE]{\thepage}
\fancyhead[LO]{\leftmark}
\fancyhead[RE]{\doctitle}

\fancypagestyle{corps}{ 
\fancyhead[RO,LE]{\thepage}
\fancyhead[LO]{\rightmark}
\fancyhead[RE]{\leftmark}
}

\renewcommand{\footrulewidth}{0pt} % pas de filet en bas
\fancypagestyle{plain}{ % pages de tetes de chapitre
\fancyhead{}
% supprime l’entete
\renewcommand{\headrulewidth}{0pt} % et le filet
}
\newcommand{\clearemptydoublepage}{%
	\newpage{\pagestyle{empty}\cleardoublepage}}


% definition des fonctions de la page de garde
\input{includes/gardedef}

% definition du titre et autres param
\def\doctitle{Naimi-Trehel extended}
\def\titre{\LARGE \doctitle}
\def\sstitre{Rapport (Mai 2012)}
\def\auteurs{
      William \textsc{Dyce} \\
      Amine \textsc{M. Elkhalsi}}
      
\def\url{https://github.com/cogitoTeam/artificial\_consciousness}

\makeglossary

\begin{document}
\include{glossary_defines}
\renewcommand{\labelitemii}{\textasteriskcentered}

\dominitoc

\thispagestyle{empty}
  \vbox to .9\vsize{%
  \vss
  \vbox to 1\vsize{%
    \haut{}{\blurb}{}
    \vfill
    \noindent\rule{\linewidth}{.5pt}
    \ligne{\vspace{1.5mm}\titre}
    \noindent\rule{\linewidth}{.5pt}
    \ligne{\normalsize{\textsc{\sstitre}}}
    \vfill
    \begin{center}
    	\includegraphics[width=0.4\textwidth]{files/logo}
    \end{center}
    \vfill
    \ligne{%
      \begin{tabular}{l}
	\vspace{5mm}
      \end{tabular}
      \begin{tabular}{c}
      \\\\
       \auteurs
      \end{tabular}
    }
    % project website
    %\ligne{%
    %\begin{tabular}{l}          
	%\vspace{15mm}
    %  \end{tabular}
    %   \texttt{\url}
    %   }
  \vss
  }
}
\clearemptydoublepage

\tableofcontents
\clearemptydoublepage

\chapter*{Introduction}

Nous tenons à remercier notre encadrant \mbox{Rodolphe} \mbox{Giroudeau}, Professeur au \mbox{\texttt{LIRMM}}\footnote{LIRMM : Laboratoire d'Informatique, de Robotique et de Micro-électronique de Montpellier.}, pour son aide, ses conseils et sa disponibilité.
Nous vous remercions également, lecteur, de lire ce rapport que nous espérons saura susciter votre intérêt.

L’objectif de ce rapport est de vous présenter notre implémentation de l'algorithme d'exclusion mutuelle de Naimi-Trehel et de l'extension tolérant aux pannes apporté par \mbox{Julien} \mbox{Sopena}, \mbox{Luciana} \mbox{Arantes}, \mbox{Marin} \mbox{Bertier} and \mbox{Pierre} \mbox{Sens} de la \mbox{\texttt{CNRS}}\footnote{CNRS : Centre Nationale de la Recherche Scientifique}.

Nous allons dans un premier temps vous faire une présentation de l’algorithme de Naimi-Trehel : son application, son fonctionnement et ses extensions. Nous aborderons ainsi l’article de recherche qui nous a servi d’appui pour la réalisation du projet. Puis nous parlerons du déroulement du projet, de notre implémentation et des problèmes rencontrés.


% ----------------------------------------------------------------------------------
% THÉORIE
% ----------------------------------------------------------------------------------
\chapter{Théorie}

\section{Exclusion mutuelle}
\subsection{Problématique}
Il est souvent question d'assurer un accès exclusive à une ressource, que ce soit matérielle ou logicielle, locale ou en réseau. Les algorithmes pour assurer un accès en tour-par-tour d'un ensemble de \og sites \fg{}\footnote{ site : processus, fil d'exécution, client ou autre agent plus ou moins abstrait.} utilisent en générale soit les permissions\footnote{protocole à permissions : le candidate demande l'accord de tous les autres sites.} soit un jeton\footnote{protocole à jeton : un objet unique transmissible donne accès à la section critique}. Nous appelons \og section critique \fg{} la partie du programme d'un site aillant besoin de cette ressource partagé.

Les protocole à jeton utilisent moins de messages en générale, vu qu'à priori il ne faut demander qu'au site qui détient le jeton. En théorie il serait possible d'obtenir le jeton en un nombre constant de message, à condition de connaitre où il se trouve à tout moment. En pratique c'est rarement le cas, mais maintenir une structure d'arbre distribué\footnote{arbre distribué : chaque site ne connait que son père.} avec le jeton en racine nous permet de le retrouver à tout moment en un nombre de messages en moyenne logarithmique en le nombre de sites. Nous passons donc de $O(n)$ à $O(log(n))$.
\subsection{Algorithme de Naimi-Trehel}
L’Algorithme de Naimi-Trehel est un algorithme distribué permettant de gérer l’accès à une ressource à une seule entrée.
Ce dernier se base sur n sites [S0, S1, S2, …, Sn-1] qui vont vouloir chacun, à des moments donnés, accéder à une ressource (ressource système, programme informatique \dots etc) qui ne peut être utilisée par deux sites en même temps.
Afin de gérer l’accès concurrent et l’exclusion mutuelle, celui-ci utilise un jeton en posant comme règle simple.
Seul celui qui possède le jeton peut entrer en section critique.
Un seul jeton existe sur tout le réseau, pas de décuplement.
Bien entendu, afin que tout le monde puisse accéder à la ressource, les sites devront pouvoir se passer le jeton entre eux en fonction des demandes.

\section{Tolérance aux pannes}
\subsection{Problématique}
Il est souvent question d'assurer un accès exclusive à une ressource, que ce soit matérielle ou logicielle, locale ou en réseau. Les algorithmes pour assurer un accès en tour-par-tour d'un ensemble de \og sites \fg{}\footnote{ site : processus, fil d'exécution, client ou autre agent plus ou moins abstrait.} utilisent en générale soit les permissions\footnote{protocole à permissions : le candidate demande l'accord de tous les autres sites.} soit un jeton\footnote{protocole à jeton : un objet unique transmissible donne accès à la section critique}. Nous appelons \og section critique \fg{} la partie du programme d'un site aillant besoin de cette ressource partagé.

Les protocole à jeton utilisent moins de messages en générale, vu qu'à priori il ne faut demander qu'au site qui détient le jeton. En théorie il serait possible d'obtenir le jeton en un nombre constant de message, à condition de connaitre où il se trouve à tout moment. En pratique c'est rarement le cas, mais maintenir une structure d'arbre distribué\footnote{arbre distribué : chaque site ne connait que son père.} avec le jeton en racine nous permet de le retrouver à tout moment en un nombre de messages en moyenne logarithmique en le nombre de sites. Nous passons donc de $O(n)$ à $O(log(n))$.
\subsection{Extension de Naimi-Trehel}
Plusieurs extensions ont étés proposés pour combler les faiblesses des algorithmes à jeton :

\begin{itemize} 
\item Nishio et al. proposent une extension fiable de l'algorithme à diffusion de Suzuki-Kasami. Pour régénérer le jeton, leur algorithme nécessite un acquittement de tous les sites. Ainsi, la défaillance d'un seul nœud retarde la régénération du jeton jusqu'à son retour. 
\item Afin d'avoir un recouvrement plus rapide Manivannan et Singhal présentent un nouvel algorithme dans lequel seuls les nœuds non fautifs doivent répondre. Cependant, leur approche n'est pas équitable devant les fautes car ils supposent que deux sites particuliers ne peuvent pas être en panne : le dernier nœud x ayant exécuté la section critique et celui à qui x à transmis le jeton.
\item Chang et al. présentent un algorithme tolérant aux fautes basé sur un arbre de requête. La fiabilité de l'arbre est assurée par l'introduction de chemins redondants et par l'utilisation d'un mécanisme d'élection. Cependant, l'ajout des chemins alternatifs et surtout le mécanisme de prévention de boucles augmentent significativement le coût de l'algorithme.
\item Naimi et Tréhel introduisent une version fiable de leur algorithme. En absence de faute, l'algorithme initial n'est pas modifié. En revanche, le recouvrement des fautes est particulièrement coûteux en termes de messages et de latence, et l'ordre des demandes est perdu : il faut complètement régénérer l'arbre et la file, ce qui nécessite de multiple diffusions. 
\item Plus récemment, Mueller a proposé aussi une version fiable de l'algorithme de Naimi-Tréhel sans utiliser de diffusions. Sa solution repose sur un anneau reliant tous les nœuds.
Les inconvénients d'une telle approche sont sa faible extensibilité et, plus encore, son impossibilité à gérer plusieurs fautes simultanées.
\end{itemize}

Cependant ne figure pas ici de solution permettant ajouter une complète fiabilité (soit N-1 pannes quelconques) sans sacrifier de complexité.
\subsection{Extension de Sopena, Arantes, Bertier et Sens}
L'idée de l'extension de \mbox{Julien} \mbox{Sopena}, \mbox{Luciana} \mbox{Arantes}, \mbox{Marin} \mbox{Bertier} et \mbox{Pierre} \mbox{Sens} est de reconstruire l'arbre et la file en recollant les morceau cassés ensembles. Pour ceci il faut passer d'une file distribué à une sorte de liste doublement-chainé partiellement-distribué. Chaque site dans la file connait déjà son successeur, mais maintenant il connaitra en plus ses $k$ prédécesseurs dans la file. Il peut alors vérifier leur bon fonctionnement en continue et :

\begin{itemize}
\item Si la chaine perd une maille le site dernière n'a que se connecter au prochain successeur.
\item Si la chaine perd $k$ mailles successives le site doit chercher le reste de la file \dots
\end{itemize}

Pour permettre cette deuxième opération nous stockons en plus la position dans la file pour chaque site, donc le nombre de mailles entre lui et le jeton. Il nous est alors possible, même si $k$ sites séquentielles tombent en panne, de trouver la bonne position dans la file. Cette reconstruction fait cependant appelle à des diffusions, donc il vaut mieux que $k$ soit assez grand pour que nous puissions éviter cette éventualité la plupart du temps.

Il est à noter que nous sacrifions en partie l'élégance de la solution de base en stockant de sous-parties de la file sur chaque site. Clairement la redondance est importante pour les systèmes distribués sujet au bruits. 
Il est également intéressant d'observer cette corrélation directe entre l'instabilité d'une société multi-agent et la manque de vue d'ensemble des éléments qui le compose. La généralisation de connaissances et non la spécialisation serait roi.

% ----------------------------------------------------------------------------------
% PRATIQUE
% ----------------------------------------------------------------------------------
\chapter{Pratique}

\section{Architecture orienté agent}
\subsection{Empilement de couches}
Nous voulions depuis le début s'abstraire de la couche de transport en créant un ensemble de fonctions d'envoie et de réception de haut niveau que nous pourrions utiliser ensuite pour implémenter les algorithmes sans se soucier de ce qui se passe \og sous le capot\fg{}. Il était claire que le deuxième étape serait de programmer l'algorithme de Naimi-Trehel et que le troisième serait d'ajouter l'extension tolérant aux pannes.

Nous somme donc arrivés à une sorte d'empilement de couches à la \texttt{ISO-OSI}\footnote{\texttt{ISO-OSI} : International Standards Organisation Open Systems Interconnection.}. Nous est paru pertinent de prendre un langage à objet, soit le $C++$, et d'implémenter chaque couche comme un héritier du couche de dessus. Un site Naimi-Trehel Tolérant est donc une extension de Site Naimi-Trehel, lui même extension de simple Site.

Le problème majeure de cette stratégie de développement est qu'il ne s'avère pas facilement parallélisable. En effet, chaque couche dépendant de celui de dessus, l'implémentation fut contrainte à suivre une séquence d'étapes bien linéaire. Ce n'est donc pas si surprenant que nous avions eu quelques difficultés pour travailler en binôme \dots
\subsection{\textsc{Site}}
\input{3_architecture_oriente_agent/2_site.tex}
\subsection{\textsc{NaimiTrehelSite}}
\input{3_architecture_oriente_agent/3_naimitrehelsite.tex}
\subsection{\textsc{SafeNaimiTrehelSite}}
\input{3_architecture_oriente_agent/4_safenaimitrehelsite.tex}

\section{Attente actif}
\subsection{Problématique}
Il est souvent question d'assurer un accès exclusive à une ressource, que ce soit matérielle ou logicielle, locale ou en réseau. Les algorithmes pour assurer un accès en tour-par-tour d'un ensemble de \og sites \fg{}\footnote{ site : processus, fil d'exécution, client ou autre agent plus ou moins abstrait.} utilisent en générale soit les permissions\footnote{protocole à permissions : le candidate demande l'accord de tous les autres sites.} soit un jeton\footnote{protocole à jeton : un objet unique transmissible donne accès à la section critique}. Nous appelons \og section critique \fg{} la partie du programme d'un site aillant besoin de cette ressource partagé.

Les protocole à jeton utilisent moins de messages en générale, vu qu'à priori il ne faut demander qu'au site qui détient le jeton. En théorie il serait possible d'obtenir le jeton en un nombre constant de message, à condition de connaitre où il se trouve à tout moment. En pratique c'est rarement le cas, mais maintenir une structure d'arbre distribué\footnote{arbre distribué : chaque site ne connait que son père.} avec le jeton en racine nous permet de le retrouver à tout moment en un nombre de messages en moyenne logarithmique en le nombre de sites. Nous passons donc de $O(n)$ à $O(log(n))$.
\subsection{Programmation concurrente}
\subsubsection{Multi-processus}
Une première idée fut d'utiliser un \texttt{fork} au moment de l'attente : nous créerons ainsi deux sous-processus, père et fils, le premier pour attendre et le deuxième pour retourner s'occuper des autres taches. Cette approche à première vu assez prometteur pose cependant de nombreux problèmes. 

La mémoire n'est pas partagé entre processus mais copié lors du branchement. De ce fait chaque processus travail sur une version locale du site, avec ses propres attributs : nous nous trouvons ainsi avec deux sites en locale, un vrai gaspillage de mémoire. C'est aussi un problème non-triviale de synchroniser le deux pour les faire communiquer : il faudrait que le processus en attente reçoive un message de son père pour lui dire que le jeton est reçu, ce qui nous mène dans exactement le même problème que celui que nous tentions résoudre de base!

\subsubsection{Multi-thread}
Les files d'exécution ont l'avantage de partager de la mémoire, ce qui permettrai aux deux d'accéder au même structure site. La bibliothèque \texttt{SDL} a en plus sa propre structure \texttt{SDL\_Thread} permettant une utilisation basique très facile à comprendre. 

Cependant nous ne nous échappons pas des problèmes de synchronisation et surtout d'exclusion mutuelle : maintenant il faut faire très attente car nous pouvions facilement avoir des accès concurrents à cette mémoire partagé. C'est en fait ironique qu'on implémentant un algorithme d'exclusion mutuelle réseau nous nous retrouvons avec des problèmes d'exclusion mutuelle locaux. Nous aurions pu, par exemple, utiliser Naimi-Trehel pour implémenter Naimi-Trehel.

\subsubsection{Conclusion}
La programmation concurrente fut finalement abandonné au profit d'une solution plus simple et moins \og fractale \fg{} : celui que nous pourrions appeler \og programmation par état \fg{}.
\subsection{Protocole \texttt{UDP}}
\input{4_attente_actif/3_udp.tex}
\section{Console non-bloquant}
\input{4_attente_actif/4_console_non_bloquant.tex}

\chapter*{Conclusion}

\end{document}
