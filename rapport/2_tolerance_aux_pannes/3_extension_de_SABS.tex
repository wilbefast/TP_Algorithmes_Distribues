L'idée de l'extension de \mbox{Julien} \mbox{Sopena}, \mbox{Luciana} \mbox{Arantes}, \mbox{Marin} \mbox{Bertier} et \mbox{Pierre} \mbox{Sens} est de reconstruire l'arbre et la file en recollant les morceau cassés ensembles. Pour ceci il faut passer d'une file distribué à une sorte de liste doublement-chainé partiellement-distribué. Chaque site dans la file connait déjà son successeur, mais maintenant il connaitra en plus ses $k$ prédécesseurs dans la file. Il peut alors vérifier leur bon fonctionnement en continue et :

\begin{itemize}
\item Si la chaine perd une maille le site dernière n'a que se connecter au prochain successeur.
\item Si la chaine perd $k$ mailles successives le site dernière devra chercher le reste de la file \dots
\end{itemize}

Pour permettre cette deuxième opération nous stockons en plus la position dans la file pour chaque successeur, donc le nombre de mailles entre lui et le jeton. Il nous est alors possible, même si $k$ sites séquentielles tombent en panne, de trouver la bonne position dans la file. Cette reconstruction fait cependant appelle à des diffusions, donc il vaut mieux que $k$ soit assez grand pour que nous puissions éviter cette éventualité la plupart du temps.

Il est à noter que nous sacrifions en partie l'élégance de la solution de base en stockant de sous-parties de la file sur chaque site. Clairement la redondance est importante pour les systèmes distribués sujet au bruits : il y aurait une corrélation directe entre l'instabilité d'une société et la manque de vue d'ensemble des agents qui le composent.