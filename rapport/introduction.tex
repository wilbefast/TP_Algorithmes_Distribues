L'objet de ce rapport est de vous présenter notre implémentation de l'algorithme d'exclusion mutuelle de Naimi-Trehel et de l'extension tolérant aux pannes apporté par \mbox{Julien} \mbox{Sopena}, \mbox{Luciana} \mbox{Arantes}, \mbox{Marin} \mbox{Bertier} and \mbox{Pierre} \mbox{Sens} de la \mbox{\texttt{CNRS}}\footnote{CNRS : Centre Nationale de la Recherche Scientifique}.

Nous allons dans un premier temps vous faire une présentation de l’algorithme en question : ses applications, son fonctionnement et ses extensions. Nous aborderons ainsi l’article de recherche qui nous a servi d’appui pour la réalisation du projet. Ensuite nous parlerons du déroulement du projet, de notre implémentation et des problèmes rencontrés.

Nous tenons à remercier notre encadrant \mbox{Rodolphe} \mbox{Giroudeau}, Professeur au \mbox{\texttt{LIRMM}}\footnote{LIRMM : Laboratoire d'Informatique, de Robotique et de Micro-électronique de Montpellier.}, pour son aide, ses conseils et sa disponibilité. Ce document a pu être réalisé grâce à 
une utilisation des entêtes \LaTeX{} de notre collègue \mbox{Thibaut} \mbox{Marmin}, à qui nous devons donc également gratitude. 

Nous vous remercions finalement vous, lecteur, de lire ce rapport que nous espérons saura susciter votre intérêt.