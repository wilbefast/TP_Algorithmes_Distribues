L'algorithme de Naimi-Trehel est un protocole à jeton qui utilise, en plus d'un arbre distribué, une file d'attente distribué\footnote{file d'attente distribué : chaque site ne connait que le site qui se trouve directement après lui dans la file.}. La racine de l'arbre correspond donc pas à celui qui détient le jeton mais plutôt à celui qui se trouve au bout de la file.

Un site voulant accéder à la ressource peut alors trouver le bout de la file en parcourant l'arbre. Le site peut alors demander à la racine qu'elle lui ajout en tant qui successeur pour devenir à son tour la racine et le bout de la file d'attente. À chaque sortie de section critique le jeton est passée le long de la file jusqu'à arriver, à un moment ou un autre (la section critique étant supposé finie), entre le mains de tout ceux qui se sont ajoutés au bout de la file.

Il faut admirer l'élégance de cette solution : aucun site n'a l'image globale de la structure à laquelle il appartient, mais émerge pourtant un système qui assure à la fois l'exclusivité de l'accès à la ressource et l'ordre des demandes, tout ça avec une complexité logarithmique.