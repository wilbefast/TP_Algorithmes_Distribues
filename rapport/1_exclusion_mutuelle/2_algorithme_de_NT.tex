L’Algorithme de Naimi-Trehel est un algorithme distribué permettant de gérer l’accès à une ressource à une seule entrée.
Ce dernier se base sur n sites [S0, S1, S2, …, Sn-1] qui vont vouloir chacun, à des moments donnés, accéder à une ressource (ressource système, programme informatique \dots etc) qui ne peut être utilisée par deux sites en même temps.
Afin de gérer l’accès concurrent et l’exclusion mutuelle, celui-ci utilise un jeton en posant comme règle simple.
Seul celui qui possède le jeton peut entrer en section critique.
Un seul jeton existe sur tout le réseau, pas de décuplement.
Bien entendu, afin que tout le monde puisse accéder à la ressource, les sites devront pouvoir se passer le jeton entre eux en fonction des demandes.