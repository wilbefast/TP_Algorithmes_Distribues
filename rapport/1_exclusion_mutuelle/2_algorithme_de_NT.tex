L'algorithme de Naimi-Trehel est un protocole à jeton qui utilise, en plus d'un arbre, une file d'attente distribué\footnote{file d'attente distribué : chaque site ne connait que le site qui se trouve directement après lui dans la file.}. La racine de l'arbre ne correspond donc plus à celui qui détient le jeton mais plutôt à celui qui se trouve au bout de la file.

Un site voulant accéder à la ressource peut alors trouver le bout de la file en parcourant l'arbre. Ce site peut alors demander à la racine qu'elle lui ajout en tant que successeur pour devenir à son tour la racine et fin de la file. À chaque sortie de section critique le jeton est passée le long de la file jusqu'à arriver, à un moment ou un autre (la section critique étant supposé finie), entre le mains de tout ceux qui s'y sont ajoutés.

En résumé : aucun site n'a l'image globale de la structure à laquelle il appartient, mais émerge pourtant un système qui assure à la fois l'exclusivité de l'accès à la ressource et l'ordre des demandes, tout en restant logarithmique en nombre de messages.