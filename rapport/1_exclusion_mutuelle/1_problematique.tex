Il est souvent question d'assurer un accès exclusive à une ressource, que ce soit matérielle ou logicielle, locale ou en réseau. Les algorithmes pour assurer un accès par tour des différents sites utilisent en générale soit les permissions\footnote{protocole à permissions : le candidate demande l'accord de tous les autres sites.} soit un jeton\footnote{protocole à jeton : un objet unique transmissible donne accès à la section critique}. Nous appelons \og section critique \fg{} la partie du programme d'un site qui utilise cette ressource partagé.

Les protocole à jeton utilisent moins de messages en générale, vu qu'à priori il ne faut demander qu'au site qui détient les racine. Tout ceci évidement à condition de connaitre où se trouve le token à tout moment : maintenir une structure d'arbre distribué\footnote{arbre distribué : chaque site ne connait que son père.} avec le jeton en racine nous permet de le retrouver à tout moment en un nombre de messages en moyenne logarithmique en le nombre de sites.