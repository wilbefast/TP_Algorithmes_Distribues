Il est souvent question d'assurer un accès exclusive à une ressource, que ce soit matérielle ou logicielle, locale ou en réseau. Les algorithmes pour assurer un accès en tour-par-tour d'un ensemble de \og sites \fg{}\footnote{ site : processus, fil d'exécution, client ou autre agent plus ou moins abstrait.} utilisent en générale soit les permissions\footnote{protocole à permissions : le candidate demande l'accord de tous les autres sites.} soit un jeton\footnote{protocole à jeton : un objet unique transmissible donne accès à la section critique}. Nous appelons \og section critique \fg{} la partie du programme d'un site aillant besoin de cette ressource partagé.

Les protocole à jeton utilisent moins de messages en générale, vu qu'à priori il ne faut demander qu'au site qui détient le jeton. En théorie il serait possible d'obtenir le jeton en un nombre constant de message, à condition de connaitre où il se trouve à tout moment. En pratique c'est rarement le cas, mais maintenir une structure d'arbre distribué\footnote{arbre distribué : chaque site ne connait que son père.} avec le jeton en racine nous permet de le retrouver à tout moment en un nombre de messages en moyenne logarithmique en le nombre de sites. Nous passons donc de $O(n)$ à $O(log(n))$.