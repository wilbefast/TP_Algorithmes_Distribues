Une fonction bijective permet de passer des identificateurs de sites à des numéros de ports à partir de \textsc{49152}, ports dites \emph{privés, éphémères ou dynamiques}. Nous pourrions alors envoyer des messages d'un site à un autre en passant de l'espace des identificateurs à celui de ports, et nous pouvions connaitre l'émetteur à partir du port en utilisant la fonction inverse. 

La bibliothèque \texttt{SDL\_net} est utilisé pour le transport : nous n'avions ici pas besoin d'un très grand nombre de fonctionnalités et la \texttt{SDL}\footnote{SDL : Simple DirectMedia Layer.} propose ceux de base à travers un API très simple d'utilisation. Nous pouvions donc ouvrir et fermer un socket \texttt{UDP}, envoyer des datagrammes et bien plus encore auquel nous n'aurions pas besoin ici. Écrit en C la \texttt{SDL} promet aussi une assez bonne performance. 