Nous voulions depuis le début s'abstraire de la couche de transport en créant un ensemble de fonctions d'envoie et de réception de haut niveau que nous pourrions utiliser ensuite pour implémenter les algorithmes sans se soucier de ce qui se passe \og sous le capot\fg{}. Ensuite il semblait évident que programmer l'algorithme de Naimi-Trehel serait le second étape avant d'ajouter l'extension tolérant aux pannes à ceci.

Nous somme donc arrivés à une sorte d'empilement de couches à la \texttt{ISO-OSI}\footnote{\texttt{ISO-OSI} : International Standards Organisation Open Systems Interconnection.}. Il nous est paru pertinent de prendre un langage à objet, soit le \texttt{C++}, et d'implémenter chaque couche comme un héritier du couche de dessus. Un site Naimi-Trehel Tolérant est donc une extension de Site Naimi-Trehel, lui même extension de simple Site.

Le problème majeure de cette stratégie de développement est qu'il ne s'avère pas facilement parallélisable\footnote{ce qui est un peu ironique étant donnée la nature du cours}. En effet, chaque couche dépendant de celui de dessus, l'implémentation fut contrainte à suivre une séquence d'étapes bien linéaire. Ce n'est donc pas si surprenant que nous avions eu quelques difficultés pour travailler en binôme. 