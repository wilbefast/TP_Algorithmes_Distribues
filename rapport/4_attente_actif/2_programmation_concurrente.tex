\subsubsection{Multi-processus}
Une première idée fut d'utiliser un \texttt{fork} au moment de l'attente : nous créerons ainsi deux sous-processus, père et fils, le premier pour attendre et le deuxième pour retourner s'occuper des autres taches. Cette approche à première vu assez prometteur pose cependant de nombreux problèmes. 

La mémoire n'est pas partagé entre processus mais copié lors du branchement. De ce fait chaque processus travail sur une version locale du site, avec ses propres attributs : nous nous trouvons ainsi avec deux sites en locale, un vrai gaspillage de mémoire. C'est aussi un problème non-triviale de synchroniser le deux pour les faire communiquer : il faudrait que le processus en attente reçoive un message de son père pour lui dire que le jeton est reçu, ce qui nous mène dans exactement le même problème que celui que nous tentions résoudre de base!

\subsubsection{Multi-thread}
Les files d'exécution ont l'avantage de partager de la mémoire, ce qui permettrai aux deux d'accéder au même structure site. La bibliothèque \texttt{SDL} a en plus sa propre structure \texttt{SDL\_Thread} permettant une utilisation basique très facile à comprendre. 

Cependant nous ne nous échappons pas des problèmes de synchronisation et surtout d'exclusion mutuelle : maintenant il faut faire très attente car nous pouvions facilement avoir des accès concurrents à cette mémoire partagé. C'est en fait ironique qu'on implémentant un algorithme d'exclusion mutuelle réseau nous nous retrouvons avec des problèmes d'exclusion mutuelle locaux. Nous aurions pu, par exemple, utiliser Naimi-Trehel pour implémenter Naimi-Trehel.

\subsubsection{Conclusion}
La programmation concurrente fut finalement abandonné au profit d'une solution plus simple et moins \og fractale \fg{} : celui que nous pourrions appeler \og programmation par état \fg{}.