Les algorithmes distribués ont très souvent besoin d'envoyer un message et d'attendre la réponse, et l'algorithme de Naimi-Trehel ne fait pas un exception. Prenons par exemple le protocole de demande de la section critique :

\begin{algorithm}[H]
  \caption{$Demander_i$()}
  \Donnees
  {\\
    i \textit{// numéro du site voulant accéder à la section critique}\\
  }
  \Deb
  {
  	$demandeur_i$ $\leftarrow$ $\perp$ \;
  	
  	\Si{$p\grave{e}re_i$ $\neq$ nil}
  	{
  		Envoyer REQUEST à $p\grave{e}re_i$ \;
  		$p\grave{e}re_i$ $\leftarrow$ nil \;
  	}
	attendre($avoir\_jeton_i$) \;
  	section\_critique() \;

  }	
\end{algorithm}

C'est très simple sur papier, mais en pratique cette function \og attendre \fg{} est non-triviale à implémenter tout simplement parce que le site de doit pas se bloquer dessus. En effet le site doit continuer à répondre aux messages qu'il reçoit pendant qu'il attend la section critique.  
