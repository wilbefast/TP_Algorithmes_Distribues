Il aurait été intriguant de pouvoir lancer une batterie de testes à fin de voir si la complexité de l'algorithme étendu respecte la borne logarithmique promis par ses concepteurs. L'idée serait de le comparer à la version, pour ainsi dire, \og de défaut \fg{} proposé par Naimi et Trehel. 

Malheureusement les testes de systèmes distribués peuvent être un peu compliqués à mettre en œuvre, sans l'utilisation d'infrastructures adaptés et nous n'avions pas sus assez bien gérer notre temps pour nous permettre la construction de notre propre plateforme de testes.

Ce que nous apportons est néanmoins une preuve palpable du fonctionnement de l'algorithme de base et de son cousin étendu. Nous voyons clairement que ce dernier résiste aux pannes et comment il se reconstruit, ce qui pourrait aider ceux que les preuves formelles n'aurait pas convaincus. Quant au Naimi-Trehel de base, rien ne faut une expérience pratique pour comprendre le fonctionnement d'un algorithme. 